\documentclass{article}
\title{Getting Started with XShaderCompiler}
\author{Lukas Hermanns}
\date{\today}

\usepackage{listings}
\usepackage{color}
\usepackage{pxfonts}
\usepackage{geometry}
\usepackage[T1]{fontenc}
\usepackage{xspace}
\usepackage{hyperref}
\usepackage{graphicx}
\usepackage{float}
\usepackage{pdfpages}

\geometry{
	a4paper,
	left=15mm,
	right=15mm,
	top=20mm,
	bottom=20mm
}

\begin{document}

\definecolor{brightBlueColor}{rgb}{0.5, 0.5, 1.0}
\definecolor{darkBlueColor}{rgb}{0.0, 0.0, 0.5}

\def\XSC{\textcolor{darkBlueColor}{XShaderCompiler}\xspace}

\lstset{
	language = C++,
	basicstyle = \footnotesize\ttfamily,
	commentstyle = \itshape\color{brightBlueColor},
	keywordstyle = \bfseries\color{darkBlueColor},
	stringstyle = \color{red},
	frame = single,
	tabsize = 4,
	showstringspaces = false,
	numbers = none,
	morekeywords = {
		% HLSL
		float2, float3, float4, float4x4,
		domain, partitioning, outputtopology, outputcontrolpoints, patchconstantfunc, maxtessfactor,
		% GLSL
		vec2, vec3, vec4, mat4,
		layout, in, out, inout,
	},
}

\maketitle

\begin{figure}[ht]
	\centering
	\includegraphics[width=0.5\textwidth]{images/Xsc_Logo.pdf}
\end{figure}

\newpage


%----------------------------------------------------------------------------------------
%	CONTENTS
%----------------------------------------------------------------------------------------

\tableofcontents

\newpage


%----------------------------------------------------------------------------------------
%	INTRRODUCTION
%----------------------------------------------------------------------------------------

\section{Introduction}

\XSC (``Cross Shader Compiler'') is a cross-compiler (also called trans-compiler),
which translates HLSL code (DirectX High Level Shading Language,
see \href{https://msdn.microsoft.com/en-us/library/windows/desktop/bb509561(v=vs.85).aspx}{msdn.microsoft.com}) 
of Shader Model 4 and 5 into GLSL code (OpenGL Shading Language,
see \href{https://www.opengl.org/wiki/OpenGL_Shading_Language}{www.opengl.org}).


%----------------------------------------------------------------------------------------
%	PROGRESS
%----------------------------------------------------------------------------------------

\newpage
\section{Progress}

This project is still in its early steps. This document is written for \XSC \textbf{Version 0.10 Alpha}.

\subsection{ToDo List}
See \href{https://github.com/LukasBanana/XShaderCompiler/blob/master/TODO.md}{TODO.md file on github.com}


%----------------------------------------------------------------------------------------
%	OFFLINE COMPILER
%----------------------------------------------------------------------------------------

\newpage
\section{Offline Compiler}

The offline compiler (named \texttt{xsc}) can be used to cross-compile your shaders without building any custom application.
It has similar commands like other common compilers (such as GCC), e.g. \texttt{-O} to enable optimization.
To show the description of all commands, type simply \texttt{xsc} or \texttt{xsc ----help} into a terminal or command line.

\subsection{Commands}

Here is a brief overview of the most important commands:
\begin{itemize}
	\item[] \textbf{\texttt{-T, ----target \textit{TARGET}}} \\
	Sets the shader target specified by \texttt{\textit{TARGET}}.
	These values are adopted from LunarG's reference compiler
	\href{https://www.khronos.org/opengles/sdk/tools/Reference-Compiler/}{\texttt{glslang}}.
	Valid values for \texttt{\textit{TARGET}} are:
	\begin{itemize}
		\item[] \textbf{\texttt{vert}} (for \textbf{Vert}ex Shader)
		\item[] \textbf{\texttt{tesc}} (for \textbf{Tes}sellation \textbf{C}ontrol Shader, also called Hull Shader)
		\item[] \textbf{\texttt{tese}} (for \textbf{Tes}sellation \textbf{E}valuation Shader, also called Domain Shader)
		\item[] \textbf{\texttt{geom}} (for \textbf{Geom}etry Shader)
		\item[] \textbf{\texttt{frag}} (for \textbf{Frag}ment Shader, also called Pixel Shader)
		\item[] \textbf{\texttt{comp}} (for \textbf{Comp}ute Shader)
	\end{itemize}
	
	\item[] \textbf{\texttt{-E, ----entry \textit{ENTRY}}} \\
	Sets the shader entry point (i.e. main function) specified by \texttt{\textit{ENTRY}}.
	
	\item[] \textbf{\texttt{-I, ----include \textit{PATH}}} \\
	Adds the file path, specified by \texttt{\textit{PATH}}, to the include search paths.
	
	\item[] \textbf{\texttt{-o, ----output \textit{FILE}}} \\
	Sets the filename of the output file specified by \texttt{\textit{FILE}}.
	The default value is ``\texttt{\textit{FILE}.\textit{ENTRY}.\textit{TARGET}}'',
	where \texttt{\textit{FILE}} is the filename of the input shader file, \texttt{\textit{ENTRY}} is the shader entry point,
	and \texttt{\textit{TARGET}} is the shader output target. The asterisk character `*' can be included to
	re-use the default value, e.g. ``\texttt{OutputFolder/*}'' will result into
	``\texttt{OutputFolder/\textit{FILE}.\textit{ENTRY}.\textit{TARGET}}''
	
	\item[] \textbf{\texttt{-Vin, ----version-in \textit{VERSION}}} \\
	Sets the input shader version specified by \texttt{\textit{VERSION}}.
	Valid values for \texttt{\textit{VERSION}} are:
	\begin{itemize}
		\item[] \textbf{\texttt{Cg}} (similar to HLSL3 with additional typenames)
		\item[] \textbf{\texttt{HLSL3}} (Shader Model 3)
		\item[] \textbf{\texttt{HLSL4}} (Shader Model 4)
		\item[] \textbf{\texttt{HLSL5}} (Shader Model 5) \textit{default value}
		\item[] \textbf{\texttt{HLSL6}} (Shader Model 6)
		\item[] \textbf{\texttt{GLSL}} (GLSL for OpenGL) \textit{only pre-processing supported}
		\item[] \textbf{\texttt{ESSL}} (GLSL for OpenGL ES) \textit{only pre-processing supported}
		\item[] \textbf{\texttt{VKSL}} (GLSL for Vulkan) \textit{only pre-processing supported}
	\end{itemize}
	
	\item[] \textbf{\texttt{-Vout, ----version-out \textit{VERSION}}} \\
	Sets the output shader version specified by \texttt{\textit{VERSION}}.
	Valid values for \texttt{\textit{VERSION}} are:
	\begin{itemize}
		\item[] \textbf{\texttt{GLSL}} (for automatic deduction of the minimal required GLSL version) \textit{default value}
		\item[] \textbf{\texttt{GLSL110}} (for GLSL 1.10) \textit{only partially supported}
		\item[] \textbf{\texttt{GLSL120}} (for GLSL 1.20) \textit{only partially supported}
		\item[] \textbf{\texttt{GLSL130}} (for GLSL 1.30)
		\item[] \textbf{\texttt{GLSL140}} (for GLSL 1.40)
		\item[] \textbf{\texttt{GLSL150}} (for GLSL 1.50)
		\item[] \textbf{\texttt{GLSL330}} (for GLSL 3.30)
		\item[] \textbf{\texttt{GLSL400}} (for GLSL 4.00)
		\item[] \textbf{\texttt{GLSL410}} (for GLSL 4.10)
		\item[] \textbf{\texttt{GLSL420}} (for GLSL 4.20)
		\item[] \textbf{\texttt{GLSL430}} (for GLSL 4.30)
		\item[] \textbf{\texttt{GLSL440}} (for GLSL 4.40)
		\item[] \textbf{\texttt{GLSL450}} (for GLSL 4.50)
	\end{itemize}
	
\end{itemize}

\subsection{Example}

Here is a small use case example. Consider the following minimal HLSL vertex shader,
stored in a file named ``\texttt{Example.hlsl}'':
\begin{lstlisting}
float4 VertexMain(float3 coord : COORD) : SV_Position
{
    return float4(coord, 1);
}
\end{lstlisting}
Now enter the following into your command prompt:

\texttt{xsc -T vert -E VertexMain Example.hlsl}
\\ \\
\noindent
The resulting GLSL shader will be stored in a file named ``\texttt{Example.VertexMain.vert}'', and looks like this:
\begin{lstlisting}
#version 130

in vec3 coord;

void main()
{
    gl_Position = vec4(coord, 1);
}
\end{lstlisting}

\subsection{Initialization File}

The offline compiler always checks for an optional initialization file named \textbf{\texttt{xsc.ini}},
which must be placed at the same location, where the executable file \texttt{xsc} is located.
Each line in this initialization file is interpreted as a single command line, without the leading \texttt{xsc} name.
Here is an example of the content of such an initialization file:
\begin{lstlisting}[caption=\texttt{xsc.ini}]
-Vout VKSL -Wall
\end{lstlisting}
This will always set the shader output to VKSL (GLSL for Vulkan) and enable all warnings.


%----------------------------------------------------------------------------------------
%	API OVERVIEW
%----------------------------------------------------------------------------------------

%\newpage
%\section{API Overview}




%----------------------------------------------------------------------------------------
%	LIMITATIONS
%----------------------------------------------------------------------------------------

\newpage
\section{Limitations}

There are several limitations for your HLSL shaders you want to translate to GLSL with the \XSC
which are described in this section.

\subsection{Tessellation Shaders}

\emph{(The translation of tessellation shaders is currently in progress but here is a brief overview
of the currently known limitations)}
\\
\\
The most tessellation attributes in HLSL are specified for the tessellation-control shader (alias ``Hull Shader''),
but a few of them are required for the tessellation-evaluation shader (alias ``Domain Shader'').
These are: \texttt{partitioning}, and \texttt{outputtopology}. Here is an example of an HLSL Tessellation Shader:
\begin{lstlisting}[title={\texttt{Example.hlsl}}]
[domain("quad")]                         // Required for Tessellation-Control (in GLSL)
[outputcontrolpoints(4)]                 // Required for Tessellation-Control (in GLSL)
[patchconstantfunc("PatchConstantFunc")] // Required for Tessellation-Control (in GLSL)
[partitioning("fractional_odd")]         // Required for Tessellation-Evaluation (in GLSL)
[outputtopology("triangle_ccw")]         // Required for Tessellation-Evaluation (in GLSL)
OutputHS HullShader(/* ... */)
{
	/* ... */
}

[domain("quad")]                         // Required for Tessellation-Evaluation (in GLSL)
OutputDS DomainShader(/* ... */)
{
	/* ... */
}
\end{lstlisting}
These attribute must be distributed into two GLSL shaders:
\begin{lstlisting}[title={\texttt{Example.HullShader.tesc}}]
layout(vertices = 4) in;
/* ... */
\end{lstlisting}
\begin{lstlisting}[title={\texttt{Example.DomainShader.tese}}]
layout(quads, fractional_odd_spacing, ccw) in;
/* ... */
\end{lstlisting}
The information for \texttt{fractional\_odd\_spacing} and \texttt{ccw} in the \texttt{Example.DomainShader.tese} shader file
are taken from the tessellation-control shader, although a tessellation-evaluation shader is written.
That means, both the tessellation-control- \emph{and} the tessellation-evaluation shaders must be contained
in the same shader source file (or at least in one of the included files) to guarantee a full translation of all information.
Otherwise default values will be used.

To specify the secondary entry point (in the above example ``\texttt{HullShader}'') use the
\texttt{secondaryEntryPoint} member in the \texttt{Xsc::ShaderInput} structure
or the ``\texttt{-E2, ----entry2 ENTRY}'' shell command.







%----------------------------------------------------------------------------------------
%	APPENDIX
%----------------------------------------------------------------------------------------

%\newpage

%\section{Appendix}

%Foo bar ...


\end{document}
